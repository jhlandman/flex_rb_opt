% Declaration ----------------------------------------------------------

\def\abstractname{Declaration}
\begin{abstract}

I declare that this dissertation is my own, unaided work. It is being submitted for the Degree of Master of Philosophy in the University of the Cape Town. It has not been submitted before for any degree or examination in any other University. \bigskip \bigskip 


\noindent \today
\end{abstract}

% Abstract -------------------------------------------------------------

\def\abstractname{Abstract}
\begin{abstract}

The purpose of this study is to present and test a general framework for risk-based investing. It permits various risk-based portfolios such as the global minimum variance, equal risk contribution and equal weight portfolios. The framework also allows for different estimation techniques to be used in finding the portfolios. The design of the study is to collate the existing research on risk-based investing, to analyse some modern methods to reduce estimation risk, to incorporate them in a single coherent framework, and to test the result with South African equity data. The techniques to reduce estimation risk draw from the usual mean-variance and risk-based optimisation literature. The techniques include regime switching, quantile regression, regularisation and subset resampling. In the South African experiment, risk-based portfolios materially outperformed the market weight portfolio out-of-sample using a Sharpe ratio measure. Additionally, the global minimum variance portfolio performed better than other risk-based portfolios. Given the long estimation window, no estimation techniques consistently outperformed the application of sample estimators only.

\noindent
\textbf{Keywords:} risk-based investing, portfolio optimisation, estimation risk.

\end{abstract}

% Acknowledgements -----------------------------------------------------

\def\abstractname{Acknowledgements}
\begin{abstract}

Professionally, I want to acknowledge my two supervisors: Emlyn for his consistent willingness to sit down and review my earlier unrefined work, and Obeid for his academic direction, keen insights, and approachable demeanour. I would also like to extend my gratitude to the JSE, who allowed me to use their data for my research. Additionally, I would like to thank David Taylor and AIFMRM for facilitating a fantastic masters programme. Finally, I want to acknowledge my mother, Charon, for her continued emotional and financial support. 

\end{abstract}

% ----------------------------------------------------------------------
